Following Bunker and Jensen \citep{Bunker} [eq. (14-6)], 
let us assume that the component of the molecular dipole moment operator along the $A$-axis 
is given by
\begin{equation}
\mu_A = \sum_j e_j R_j,
\end{equation}
where $e_j$ and $R_j$ are the charge and coordinates of the $j$th particle in the molecule,
and $j$ runs over all nuclei and electrons; $A = X, Y,$ or $Z$ and 
$(X, Y, Z)$ is a coordinate system space fixed at the 
nuclear % molecular 
centre of mass.
Then the line strength for a neutral molecule is
\begin{equation}
S_{(f \leftarrow i)} = \sum_{\Phi' \Phi} \sum_{A=X, Y, Z}    %\sum_{A=\xi, \eta, \zeta} 
\vert \langle \Phi' \vert \mu_A \vert \Phi \rangle \vert^2
\end{equation}
where the first sum runs over all (degenerate) eigenfunctions $\Phi'$ and $\Phi$ with energies 
$E'$ and $E$ respectively.
In practise, ro-vibrational eigenfunctions are obtained in molecule-fixed axis system
and therefore we need to transform the dipole moment operators into the molecule-fixed
coordinate system. This can be compactly given by Bunker and Jensen \citep{Bunker} [eq. (14-15)]
\begin{equation}
\mu^{(1,\sigma)}_s = \sum^1_{\sigma''=-1} \left[ D^{(1)}_{\sigma \sigma''}(\phi, \theta, \chi)\right]^*
\mu^{(1, \sigma'')}_m
\end{equation}
where the space fixed components of the dipole moment are given by
\begin{equation}
\mu^{(1,\pm 1)}_s = \left[ \mp \mu_X - i \mu_Y \right]/\sqrt{2}
\end{equation}
\begin{equation}
\mu^{(1,0)}_s = \mu_Z
\end{equation}
and similarly the molecule fixed components are 
\begin{equation}
\mu^{(1,\pm 1)}_m = \left[ \mp \mu_x - i \mu_y \right]/\sqrt{2}
\end{equation}
\begin{equation}
\mu^{(1,0)}_m = \mu_z
\end{equation}
Then the line strength equation above becomes
\begin{equation}
S_{(f \leftarrow i)} = \sum_{\Phi' \Phi} \sum^{1}_{\sigma = -1}
\vert \langle \Phi' \vert \mu^{(1,\sigma)}_s \vert \Phi \rangle \vert^2
= \sum_{\Phi' \Phi} \sum^{1}_{\sigma = -1} 
\left\vert 
\sum^{1}_{\sigma'' = -1}  
\langle \Phi' \vert D^{(1)*}_{\sigma \sigma''}
\mu^{(1,\sigma'')}_m \vert \Phi \rangle \right\vert^2
\end{equation}

The basis functions used by \textsc{wavr4} \citep{Tennyson}
are a direct product of radial and angular functions. 
The angular functions are non-direct product and include symmetric top functions.
Thus the eigenfunctions obtained in the process of Hamiltonian diagonalization
are linear combinations of basis functions 
\begin{equation}
\vert \Phi \rangle = \sum_{i, i_1, i_2, i_3} a_{i, i_1, i_2, i_3} 
\vert i \rangle \vert i_1 \rangle \vert i_2 \rangle \vert i_3 \rangle
\end{equation}
where $\vert i_1 \rangle$, $\vert i_2 \rangle$ and $\vert i_3 \rangle$ are radial basis functions
(they are essentially radial grid points) and the angular function is given by
\begin{equation}
\vert i \rangle = N_{Kk} \kappa^k P^{\vert k-\kappa K \vert}_j
\left[ Y^{\kappa k}_l \vert J, K, M\rangle + (-1)^{J+p+K+k} Y^{-\kappa k}_l \vert J, -K, M \rangle \right]
\end{equation}
In the last equation, $N_{Kk}$ is normalisation coefficient $1/\sqrt{2 (1+ \delta_{K0} \delta_{k0})}$,
$\kappa$ is an auxiliary (quantum) number taking values $+1$ or $-1$,
$p$ is parity quantum number taking values 0 or 1,
$j$, $l$ and $k$ are angular momentum quantum numbers, and
$\vert J, K, M\rangle$ is symmetric top wave functions.
Note that $i$ is assumed to be an aggregate quantum number running over all possible combinations
of $j$, $l$, $k$, $\kappa$, $K$, and $M$ ($p$ and $J$ are strict quantum numbers and therefore 
are the same for the whole eigenfunction).
Then the expectation value of a dipole operator is given by
\begin{equation}
\langle \Phi' \vert D^{(1)*}_{\sigma \sigma''}
\mu^{(1,\sigma'')}_m \vert \Phi \rangle
= \sum_{i, i'} \sum_{i_1, i_2, i_3} 
a_{i', i_1', i_2', i_3'} \; a_{i, i_1, i_2, i_3} 
\langle i' \vert D^{(1)*}_{\sigma \sigma''} \mu^{(1,\sigma'')}_m \vert i \rangle
\end{equation}
where we used an approximation that
$\langle i_1' \vert \langle i_2' \vert \langle i_3' \vert
\mu^{(1,\sigma'')}_m
\vert i_1 \rangle \vert i_2 \rangle \vert i_3 \rangle = 
\mu^{(1,\sigma'')}_m \delta_{i_1 i_1'} \delta_{i_2 i_2'} \delta_{i_3 i_3'}$.
That is we used the diagonal approximation for the dipole moment function in the grid basis.
It means the integration over the radial basis is reduced to taking the values of the dipole function
at the grid points. 
After substituting the expression for $\vert i \rangle$ we obtain
\begin{eqnarray} \nonumber &&
\langle \Phi' \vert D^{(1)*}_{\sigma \sigma''}
\mu^{(1,\sigma'')}_m \vert \Phi \rangle
= \sum_{i, i'} \sum_{i_1, i_2, i_3} 
a_{i', i_1', i_2', i_3'} \; a_{i, i_1, i_2, i_3} 
N_{K'k'} (\kappa')^{k'} N_{Kk} (\kappa)^{k} \int \int d\Omega \;
P^{\vert k'- \kappa' K' \vert}_{j'}
P^{\vert k - \kappa  K  \vert}_{j} 
\\ \nonumber &&
 \left[ 
 Y^{* \kappa' k'}_{l'} \mu^{(1,\sigma'')}_m Y^{\kappa k}_{l} \langle J',K',M' \vert D^{(1)*}_{\sigma \sigma''} \vert J,K,M \rangle +
\right.  
\\ \nonumber &&
 + (-1)^{J'+p'+K'+k'} \;
 Y^{* - \kappa' k'}_{l'} \mu^{(1,\sigma'')}_m Y^{\kappa k}_{l} \langle J',-K',M' \vert D^{(1)*}_{\sigma \sigma''} \vert J,K,M \rangle +
\\ \nonumber &&
 + (-1)^{J+p+K+k} \;
 Y^{* \kappa' k'}_{l'} \mu^{(1,\sigma'')}_m Y^{-\kappa k}_{l} \langle J',K',M' \vert D^{(1)*}_{\sigma \sigma''} \vert J,-K,M \rangle +
\\           &&
\left. + (-1)^{J'+p'+K'+k'+J+p+K+k} \;
 Y^{* -\kappa' k'}_{l'} \mu^{(1,\sigma'')}_m Y^{-\kappa k}_{l} \langle J',-K',M' \vert D^{(1)*}_{\sigma \sigma''} \vert J,-K,M \rangle
 \right]
\end{eqnarray}
where $\mu^{(1,\sigma'')}_m$ is assumed to depend only on the internal angular coordinates and 
the radial coordinates take the values of the radial grid points.


Taking into account eq. (14-23) \cite{Bunker}
\begin{equation}
\langle J',K',M' \vert D^{(1)*}_{\sigma \sigma''} \vert J,K,M \rangle =
(-1)^{K'+M'} \sqrt{(2J+1)(2J'+1)} 
\left( \begin{array}{ccc} 
J &        1 & J' \\
K & \sigma'' & -K'  \end{array} \right)
\left( \begin{array}{ccc} 
J &      1 & J' \\
M & \sigma & -M'  \end{array} \right)
\end{equation}
the expression in square brackets above becomes
\begin{eqnarray}  \nonumber &&
[...] = (-1)^{M'} \sqrt{(2J+1)(2J'+1)} 
\left( \begin{array}{ccc} 
J &      1 & J' \\
M & \sigma & -M'  \end{array} \right)
\\ \nonumber &&
 \left\{
 Y^{* \kappa' k'}_{l'} \mu^{(1,\sigma'')}_m Y^{\kappa k}_{l} 
\left( \begin{array}{ccc} 
J &        1 & J' \\
K & \sigma'' & -K'  \end{array} \right) (-1)^{K'} +
\right.
\\ \nonumber &&
 + Y^{* - \kappa' k'}_{l'} \mu^{(1,\sigma'')}_m Y^{\kappa k}_{l} 
\left( \begin{array}{ccc} 
J &        1 & J' \\
K & \sigma'' & K'  \end{array} \right) (-1)^{J'+p'+K'+k'} \; (-1)^{-K'} +
\\ \nonumber &&
 + Y^{* \kappa' k'}_{l'} \mu^{(1,\sigma'')}_m Y^{-\kappa k}_{l} 
\left( \begin{array}{ccc} 
 J &        1 & J' \\
-K & \sigma'' & -K'  \end{array} \right) (-1)^{J+p+K+k} \; (-1)^{K'} +
\\           &&
\left. + Y^{* -\kappa' k'}_{l'} \mu^{(1,\sigma'')}_m Y^{-\kappa k}_{l} 
\left( \begin{array}{ccc} 
 J &        1 & J' \\
-K & \sigma'' & K'  \end{array} \right) (-1)^{J'+p'+K'+k'+J+p+K+k} \; (-1)^{-K'} \right\}
\end{eqnarray}
Clearly if $a$ is an integer then $(-1)^{a} = (-1)^{-a}$. 
Since $K$ in our case is an integer $(-1)^{K'} = (-1)^{-K'}$ and 
therefore $(-1)^{K'}$ can be taken away as a common factor.


Our spherical harmonics are defined as in eq. (31) of Mladenovi{\'c} \cite{M.Mladenovic}
\begin{equation}
Y^{m}_{l} =  (-1)^{ \frac{m+\vert m \vert}{2} } P^{m}_{l} \frac{e^{i m \chi} }{\sqrt{2 \pi} }
\end{equation}
where $P^{m}_{l} (\cos \theta)$ are normalised associated Legendre functions.
They have a property that $P^{m}_{l} = P^{-m}_{l}$ and it can be shown that 
\begin{equation}
(Y^{m}_{l})^{*} = (-1)^{m} Y^{-m}_{l}
\end{equation}
Since $\kappa$ is equal to $-1$ or 1 and $k$ is not negative
\begin{equation}
Y^{\kappa k}_{l} = (-1)^{\frac{\kappa k + \vert \kappa k \vert}{2}} P^{k}_{l} \frac{e^{i \kappa k \chi} }{\sqrt{2 \pi} } = 
(-1)^{\frac{\kappa+1}{2} k } P^{k}_{l} \frac{e^{i \kappa k \chi} }{\sqrt{2 \pi} }
\end{equation}


With all that in place, the expectation value of the dipole moment function is
\begin{eqnarray}  \nonumber &&
\langle \Phi' \vert \mu^{(1,\sigma)}_s \vert \Phi \rangle = 
\sum_{i',i,i_1,i_2,i_3} a_{i',i_1,i_2,i_3} a_{i,i_1,i_2,i_3} N_{K'k'} (\kappa')^{k'} N_{Kk} (\kappa)^{k}
\\ \nonumber &&
\int \int d\Omega \; P^{\vert k' - \kappa' K' \vert}_{j'} P^{\vert k - \kappa K \vert}_{j}
(-1)^{M'+K'} \sqrt{(2 J+1)(2 J'+1)} 
\left( \begin{array}{ccc} 
 J &      1 &  J' \\
 M & \sigma & -M'  \end{array} \right)
\\ \nonumber &&
\sum^{1}_{\sigma'' = -1} \mu^{(1,\sigma'')}_m
\left[ Y^{- \kappa' k'}_{l'}  Y^{\kappa k}_{l} 
\left( \begin{array}{ccc} 
 J &        1 &  J' \\
 K & \sigma'' & -K'  \end{array} \right) (-1)^{\kappa' k'} \right. +
\\ \nonumber &&
+ Y^{ \kappa' k'}_{l'}  Y^{\kappa k}_{l} 
\left( \begin{array}{ccc} 
 J &        1 &  J' \\
 K & \sigma'' &  K'  \end{array} \right) (-1)^{\kappa' k' + J'+p'+K'+k'} +
\\ \nonumber &&
+ Y^{- \kappa' k'}_{l'}  Y^{- \kappa k}_{l} 
\left( \begin{array}{ccc} 
 J &        1 &  J' \\
-K & \sigma'' & -K'  \end{array} \right) (-1)^{\kappa' k' + J+p+K+k} +
\\           &&
\left. + Y^{ \kappa' k'}_{l'}  Y^{- \kappa k}_{l} 
\left( \begin{array}{ccc} 
 J &        1 &  J' \\
-K & \sigma'' &  K'  \end{array} \right) (-1)^{\kappa' k' + J'+p'+K'+k' + J+p+K+k} \right]
\end{eqnarray}
Using the spherical harmonics definition above we obtain
\begin{eqnarray} \nonumber &&
\langle \Phi' \vert \mu^{(1,\sigma)}_s \vert \Phi \rangle = 
\sum_{i',i,i_1,i_2,i_3} a_{i',i_1,i_2,i_3} a_{i,i_1,i_2,i_3} N_{K'k'} (\kappa')^{k'} N_{Kk} (\kappa)^{k}
\\ \nonumber &&
\int \int d\Omega \; P^{\vert k' - \kappa' K' \vert}_{j'} P^{\vert k - \kappa K \vert}_{j}
(-1)^{M'+K'} \sqrt{(2 J+1)(2 J'+1)} 
\left( \begin{array}{ccc} 
 J &      1 &  J' \\
 M & \sigma & -M'  \end{array} \right)
\\ \nonumber &&
\sum^{1}_{\sigma'' = -1} \mu^{(1,\sigma'')}_m \frac{1}{2 \pi}
\left[ P^{k'}_{l'}  P^{k}_{l} \; e^{i(- \kappa' k'+ \kappa k) \chi  }
\left( \begin{array}{ccc} 
 J &        1 &  J' \\
 K & \sigma'' & -K'  \end{array} \right) (-1)^{ \frac{- \kappa' +1}{2} k' + \kappa' k' + \frac{ \kappa +1}{2} k } \right. +
\\ \nonumber &&
+ P^{k'}_{l'}  P^{k}_{l} \; e^{i( \kappa' k'+ \kappa k) \chi  }
\left( \begin{array}{ccc} 
 J &        1 &  J' \\
 K & \sigma'' &  K'  \end{array} \right) (-1)^{\frac{\kappa' +1}{2} k' + \kappa' k' + \frac{ \kappa +1}{2} k + J'+p'+K'+k'} +
\\ \nonumber &&
+ P^{k'}_{l'}  P^{k}_{l} \; e^{i(- \kappa' k' - \kappa k) \chi  }
\left( \begin{array}{ccc} 
 J &        1 &  J' \\
-K & \sigma'' & -K'  \end{array} \right) (-1)^{\frac{- \kappa' +1}{2} k' + \kappa' k' + \frac{- \kappa +1}{2} k + J+p+K+k} +
\\           &&
\left. + P^{k'}_{l'}  P^{k}_{l} \; e^{i( \kappa' k' - \kappa k) \chi  }
\left( \begin{array}{ccc} 
 J &        1 &  J' \\
-K & \sigma'' &  K'  \end{array} \right) (-1)^{\frac{ \kappa' +1}{2} k' + \kappa' k' + \frac{- \kappa +1}{2} k + J'+p'+K'+k' + J+p+K+k} \right]
\end{eqnarray}
Since $\kappa$ is an integer the following holds 
\begin{equation}
(-1)^{\frac{4 \kappa}{2} } = 1  \; => \; (-1)^{\frac{3 \kappa}{2} } = (-1)^{\frac{- \kappa}{2} }
\end{equation}
and in particular
\begin{equation}
(-1)^{\frac{4}{2} } = 1  \; => \; (-1)^{\frac{3}{2} } = (-1)^{\frac{-1}{2} }
\end{equation}

Removing the obvious common factors and simplifying the result we obtain
\begin{eqnarray}  \nonumber &&
\langle \Phi' \vert \mu^{(1,\sigma)}_s \vert \Phi \rangle = 
(-1)^{M'} \sqrt{(2 J+1)(2 J'+1)} 
\left( \begin{array}{ccc} 
 J &      1 &  J' \\
 M & \sigma & -M'  \end{array} \right)
\\ \nonumber &&
\sum_{i',i,i_1,i_2,i_3} a_{i',i_1,i_2,i_3} a_{i,i_1,i_2,i_3} N_{K'k'} (\kappa')^{k'} N_{Kk} (\kappa)^{k} (-1)^{K'}
\\ \nonumber &&
\int \int d\Omega \; P^{\vert k' - \kappa' K' \vert}_{j'} P^{\vert k - \kappa K \vert}_{j} 
P^{k'}_{l'}  P^{k}_{l} \frac{1}{2 \pi}
\sum^{1}_{\sigma'' = -1} \mu^{(1,\sigma'')}_m 
\\ \nonumber &&
\left[ 
e^{i(- \kappa' k'+ \kappa k) \chi  }
\left( \begin{array}{ccc} 
 J &        1 &  J' \\
 K & \sigma'' & -K'  \end{array} \right) (-1)^{ \frac{\kappa' +1}{2} k' + \frac{\kappa +1}{2} k } \right. +
\\ \nonumber &&
+ e^{i( \kappa' k'+ \kappa k) \chi  }
\left( \begin{array}{ccc} 
 J &        1 &  J' \\
 K & \sigma'' &  K'  \end{array} \right) (-1)^{\frac{\kappa' +1}{2} k' + \frac{\kappa +1}{2} k + J'+p'+K'} +
\\ \nonumber &&
+ e^{i(- \kappa' k' - \kappa k) \chi  }
\left( \begin{array}{ccc} 
 J &        1 &  J' \\
-K & \sigma'' & -K'  \end{array} \right) (-1)^{\frac{\kappa' +1}{2} k' - \frac{\kappa +1}{2} k + J+p+K} +
\\           &&
\left. + e^{i( \kappa' k' - \kappa k) \chi  }
\left( \begin{array}{ccc} 
 J &        1 &  J' \\
-K & \sigma'' &  K'  \end{array} \right) (-1)^{\frac{\kappa' +1}{2} k' - \frac{\kappa +1}{2} k + J'+p'+K' + J+p+K} \right]
\end{eqnarray}
As we can see there is another common factor $(-1)^{\frac{\kappa' +1}{2} k' + \frac{ \kappa +1}{2} k}$ because
$(-1)^{(\kappa +1) k} = 1$ and therefore $(-1)^{\frac{\kappa +1}{2} k} = (-1)^{-\frac{\kappa +1}{2} k}$.


Next, we expand the sum over $\sigma''$ explicitly
\begin{eqnarray}  \nonumber &&
\langle \Phi' \vert \mu^{(1,\sigma)}_s \vert \Phi \rangle = 
(-1)^{M'} \sqrt{(2 J+1)(2 J'+1)} 
\left( \begin{array}{ccc} 
 J &      1 &  J' \\
 M & \sigma & -M'  \end{array} \right)
\\ \nonumber &&
\sum_{i',i,i_1,i_2,i_3} a_{i',i_1,i_2,i_3} a_{i,i_1,i_2,i_3} N_{K'k'} (\kappa')^{k'} N_{Kk} (\kappa)^{k} 
(-1)^{K' + \frac{\kappa' +1}{2} k' + \frac{\kappa +1}{2} k } \frac{1}{2 \pi}
\\ \nonumber &&
\int \int d\Omega \; P^{\vert k' - \kappa' K' \vert}_{j'} P^{\vert k - \kappa K \vert}_{j} 
P^{k'}_{l'}  P^{k}_{l}
\\ \nonumber &&
\left\{ e^{i(- \kappa' k'+ \kappa k) \chi  }
\left[ 
\mu^{(1,0)}_m 
\left( \begin{array}{ccc} 
 J &  1 &  J' \\
 K &  0 & -K'  \end{array} \right) +
\mu^{(1,-1)}_m 
\left( \begin{array}{ccc} 
 J &  1 &  J' \\
 K & -1 & -K'  \end{array} \right) +
\mu^{(1,1)}_m 
\left( \begin{array}{ccc} 
 J &  1 &  J' \\
 K &  1 & -K'  \end{array} \right) \right] \right. +
\\ \nonumber &&
+ e^{i( \kappa' k'+ \kappa k) \chi  } (-1)^{J'+p'+K'} 
\left[ 
\mu^{(1,0)}_m 
\left( \begin{array}{ccc} 
 J &  1 &  J' \\
 K &  0 &  K'  \end{array} \right) +
\mu^{(1,-1)}_m 
\left( \begin{array}{ccc} 
 J &  1 &  J' \\
 K & -1 &  K'  \end{array} \right) +
\mu^{(1,1)}_m 
\left( \begin{array}{ccc} 
 J &  1 &  J' \\
 K &  1 &  K'  \end{array} \right) \right] +
\\ \nonumber &&
+ e^{i(- \kappa' k' - \kappa k) \chi  } (-1)^{J+p+K}
\left[ 
\mu^{(1,0)}_m 
\left( \begin{array}{ccc} 
 J &  1 &  J' \\
-K &  0 & -K'  \end{array} \right) +
\mu^{(1,-1)}_m 
\left( \begin{array}{ccc} 
 J &  1 &  J' \\
-K & -1 & -K'  \end{array} \right) +
\mu^{(1,1)}_m 
\left( \begin{array}{ccc} 
 J &  1 &  J' \\
-K &  1 & -K'  \end{array} \right) \right] +
\\           &&
\left. + e^{i( \kappa' k' - \kappa k) \chi  } (-1)^{J'+p'+K' + J+p+K}
\left[ 
\mu^{(1,0)}_m 
\left( \begin{array}{ccc} 
 J &  1 &  J' \\
-K &  0 &  K'  \end{array} \right) +
\mu^{(1,-1)}_m 
\left( \begin{array}{ccc} 
 J &  1 &  J' \\
-K & -1 &  K'  \end{array} \right) +
\mu^{(1,1)}_m 
\left( \begin{array}{ccc} 
 J &  1 &  J' \\
-K &  1 &  K'  \end{array} \right) \right] \right\}
\end{eqnarray}

Now we can write the expectation value of the dipole moment function in a compact form
\begin{equation}
\langle \Phi' \vert \mu^{(1,\sigma)}_s \vert \Phi \rangle = 
(-1)^{M'} \sqrt{(2 J+1)(2 J'+1)} 
\left( \begin{array}{ccc} 
 J &      1 &  J' \\
 M & \sigma & -M'  \end{array} \right) \; X(J,J')
\end{equation}
where $X(J,J')$ does not depend on $M$, $M'$ and $\sigma$. Therefore the line strength is
\begin{equation}
S_{(f \leftarrow i)} = \sum_{M,M',\sigma} \vert \langle \Phi' \vert \mu^{(1,\sigma)}_s \vert \Phi \rangle \vert^2 = 
(2 J+1)(2 J'+1) \; 
\vert X^2(J,J') \vert^2 \; 
\sum^{1}_{\sigma = -1} \sum_{M}
\left( \begin{array}{ccc} 
 J &      1 &  J' \\
 M & \sigma & -(M+\sigma)  \end{array} \right)^2
\end{equation}
Since $J'$ can only be equal to $J$ or $J\pm 1$, let us consider three different cases: 
\begin{itemize}
\item[Case I:] $J' = J \leftarrow J$ i.e. $\Delta J = 0$. There are three sub-cases:\\
$\sigma = 0$ hence $M = -J ... J$, \\
$\sigma = +1$ hence $M = -J ... J-1$, \\
$\sigma = -1$ hence $M = -J+1 ... J$ \\
and in any case $\sum^{1}_{\sigma = -1} \sum_{M} (...)^2 = 1$. Therefore $S = (2J+1)^2 X^2(J,J)$.
\item[Case II:] $J' = J+1 \leftarrow J$ i.e. $\Delta J = 1$. There are also three sub-cases:\\
$\sigma = 0$ hence $M = -J ... J$, \\
$\sigma = +1$ hence $M = -J ... J$, \\
$\sigma = -1$ hence $M = -J ... J$ \\
and again in any case $\sum^{1}_{\sigma = -1} \sum_{M} (...)^2 = 1$. Therefore $S = (2J+1)(2J+3) X^2(J,J')$.
\item[Case III:] $J' = J-1 \leftarrow J$ i.e. $\Delta J = -1$. This case must be the same as Case II by symmetry (reversibility).
\end{itemize}

Using the following property of $3J$-symbols
\begin{equation}
\left( \begin{array}{ccc} 
 j_1 & j_2 & j \\
 m_1 & m_2 & m  \end{array} \right) = (-1)^{j_1+j_2+j}
\left( \begin{array}{ccc} 
 j_1 &  j_2 &  j \\
-m_1 & -m_2 & -m  \end{array} \right) 
\end{equation}
we can simplify the line strength expression to obtain
\begin{equation}
S_{(f \leftarrow i)} = (2J+1)(2J'+1) \vert X(J,J') \vert^2, \;\;\; J' = J-1, J, J+1
\end{equation}
where 
\begin{eqnarray} \nonumber &&
X(J,J') = 
\sum_{i',i,i_1,i_2,i_3} a_{i',i_1,i_2,i_3} a_{i,i_1,i_2,i_3} N_{K'k'} (\kappa')^{k'} N_{Kk} (\kappa)^{k} 
(-1)^{K' + \frac{\kappa' +1}{2} k' + \frac{\kappa +1}{2} k } \frac{1}{2 \pi}
\\ \nonumber &&
\int \int d\Omega \; P^{\vert k' - \kappa' K' \vert}_{j'} P^{\vert k - \kappa K \vert}_{j} 
P^{k'}_{l'}  P^{k}_{l}
\\ \nonumber &&
\left\{ \mu^{(1,0)}_m
\left(  e^{ i( \kappa' k'+ \kappa k) \chi  } (-1)^{J'+p'+K'} + 
        e^{-i( \kappa' k'+ \kappa k) \chi  } (-1)^{J +p +K +J + 1 +J'}  \right)
\left( \begin{array}{ccc} 
 J &  1 &  J' \\
 K &  0 &  K'  \end{array} \right) \right. +
\\ \nonumber &&
+ \mu^{(1,0)}_m 
  \left(  e^{ i(-\kappa' k'+ \kappa k) \chi  } + 
          e^{-i(-\kappa' k'+ \kappa k) \chi  } (-1)^{J'+p'+K'+ J +p +K +J + 1 +J'}  \right)
\left( \begin{array}{ccc} 
 J &  1 &  J' \\
 K &  0 & -K'  \end{array} \right) +
\\ \nonumber &&
+ \left(  \mu^{(1, 1)}_m  e^{ i( \kappa' k'- \kappa k) \chi  } (-1)^{J'+p'+K' + J +p +K} + 
          \mu^{(1,-1)}_m  e^{-i( \kappa' k'+ \kappa k) \chi  } (-1)^{J+ 1 +J'}  \right)
\left( \begin{array}{ccc} 
 J &  1 &  J' \\
-K &  1 &  K'  \end{array} \right) +
\\ \nonumber &&
+ \left(  \mu^{(1, 1)}_m  e^{-i( \kappa' k'+ \kappa k) \chi  } (-1)^{J +p +K} + 
          \mu^{(1,-1)}_m  e^{ i( \kappa' k'+ \kappa k) \chi  } (-1)^{J'+p'+K' + J+ 1 +J'}  \right)
\left( \begin{array}{ccc} 
 J &  1 &  J' \\
-K &  1 & -K'  \end{array} \right) +
\\ \nonumber &&
+ \left(  \mu^{(1, 1)}_m  e^{ i( \kappa' k'+ \kappa k) \chi  } (-1)^{J'+p'+K'} + 
          \mu^{(1,-1)}_m  e^{ i(-\kappa' k'+ \kappa k) \chi  } (-1)^{J +p +K  + J+ 1 +J'}  \right)
\left( \begin{array}{ccc} 
 J &  1 &  J' \\
 K &  1 &  K'  \end{array} \right) +
\\           &&
\left.
+ \left(  \mu^{(1, 1)}_m  e^{ i(-\kappa' k'+ \kappa k) \chi  } + 
          \mu^{(1,-1)}_m  e^{ i(-\kappa' k'+ \kappa k) \chi  } (-1)^{J'+p'+K'+ J +p +K  + J+ 1 +J'}  \right)
\left( \begin{array}{ccc} 
 J &  1 &  J' \\
 K &  1 & -K'  \end{array} \right) \right\}
\end{eqnarray}
Due to the properties of $3J$-symbols the terms in the sum above exist only under special conditions.
%the first term in the sum is only non-zero if $K'=K=0$, 
%the second if $K'=K$,
%the third if $K'=K-1$ (hence $K=1,2,3...$), 
%the fourth if $K'=-K+1$ (hence $K=0$ or 1), 
%the fifth if $K'=-K-1$ (therefore this case can not exist), and
%the sixth if $K'=K+1$ (hence $K=0,1,2...$).
Furthermore, rigorous selection rules allow only transitions between states with different parity i.e.\
$p' \ne p$ (e.g.\ see Bunker and Jensen \citep{Bunker} p.\ 417).
This can be also verified explicitly using the expressions above and by taking into account 
that the action of inversion symmetry does not change $\mu_x$ and $mu_z$ but changes the sign of $\mu_y$.
Now let us consider the above terms in more details.

Term I exists only if $K' = K = 0$
\begin{eqnarray}  \nonumber
&& \{...\} = (-1)^{J'+p'}
\left( \begin{array}{ccc} 
 J &  1 &  J' \\
 0 &  0 &  0   \end{array} \right)
\mu_z \times \\
&&
\left[ \cos ( \kappa' k' + \kappa k) \chi + i \sin ( \kappa' k' + \kappa k) \chi + 
\left( \cos ( \kappa' k' + \kappa k) \chi - i \sin ( \kappa' k' + \kappa k) \chi \right) (-1)^{p'+p+1} \right]
\end{eqnarray}
Since $p' \ne p$ then $(-1)^{p'+p+1}=1$ and
\begin{equation}
\{...\} = (-1)^{J'+p'}
\left( \begin{array}{ccc} 
 J &  1 &  J' \\
 0 &  0 &  0   \end{array} \right)
\mu_z \; 2 \cos ( \kappa' k' + \kappa k) \chi
\end{equation}


Term II exsits only if $K' = K$
\begin{eqnarray} \nonumber
&& \{...\} = 
\left( \begin{array}{ccc} 
 J &  1 &  J' \\
 K &  0 & -K  \end{array} \right)
\mu_z \times \\
&&
\left[ \cos ( -\kappa' k' + \kappa k) \chi + i \sin ( -\kappa' k' + \kappa k) \chi + 
\left( \cos ( -\kappa' k' + \kappa k) \chi - i \sin ( -\kappa' k' + \kappa k) \chi \right) (-1)^{p'+p+1} \right]
\end{eqnarray}
Since $p' \ne p$ then $(-1)^{p'+p+1}=1$ and
\begin{equation}
\{...\} =     
\left( \begin{array}{ccc} 
 J &  1 &  J' \\
 K &  0 & -K   \end{array} \right)
\mu_z \; 2 \cos (-\kappa' k' + \kappa k) \chi
\end{equation}


Term III exists only if $K' = K -1$ i.e. $(-1)^{K'+K} = 1$, and $K = 1, 2, 3...$ (while $K' = 0, 1, 2...$ respectively)
\begin{eqnarray}  \nonumber
&& \{...\} = 
\left( \begin{array}{ccc} 
 J &  1 &  J' \\
-K &  1 & K-1  \end{array} \right)
\frac{(-1)^{J'+J+1}}{\sqrt{2}} \\ \nonumber
&&
\left[ ( -\mu_x - i \mu_y) 
\left( \cos ( \kappa' k' - \kappa k) \chi + i \sin ( \kappa' k' - \kappa k) \chi \right) (-1)^{p'+p} + \right. \\
&& \left.
+ ( \mu_x - i \mu_y)
\left( \cos ( \kappa' k' - \kappa k) \chi - i \sin ( \kappa' k' - \kappa k) \chi \right) \right] \\ \nonumber
&& = 
\left( \begin{array}{ccc} 
 J &  1 &  J' \\
-K &  1 & K-1  \end{array} \right)
\frac{(-1)^{J'+J+1}}{\sqrt{2}} \\ \nonumber
&& \left[ 
-i \mu_y \cos ( \kappa' k' - \kappa k) \chi \left( (-1)^{p'+p} + 1 \right)
-  \mu_x \cos ( \kappa' k' - \kappa k) \chi \left( (-1)^{p'+p} - 1 \right) - \right. \\
&& \left.
-i \mu_x \sin ( \kappa' k' - \kappa k) \chi \left( (-1)^{p'+p} + 1 \right)
+  \mu_y \sin ( \kappa' k' - \kappa k) \chi \left( (-1)^{p'+p} - 1 \right) \right]
\end{eqnarray}
Since $p' \ne p$ then $(-1)^{p'+p} = -1$ and
\begin{equation}
\{...\} = 
\left( \begin{array}{ccc} 
 J &  1 &  J' \\
-K &  1 & K-1  \end{array} \right)
\frac{(-1)^{J'+J}}{\sqrt{2}} 2  \left[
\mu_y \sin ( \kappa' k' - \kappa k) \chi -  \mu_x \cos ( \kappa' k' - \kappa k) \chi \right]
\end{equation}

Term IV exists only if $K' = -K +1$ hence either $K = 0$ and $K'=1$ or $K = 1$ and $K'=0$
\begin{eqnarray} \nonumber
&& \{...\} = 
\left( \begin{array}{ccc} 
 J &  1 &  J' \\
-K &  1 & K-1  \end{array} \right)
\frac{(-1)^{J+p+K}}{\sqrt{2}} \\ \nonumber
&& \left[ ( -\mu_x - i \mu_y) 
\left( \cos ( \kappa' k' + \kappa k) \chi - i \sin ( \kappa' k' + \kappa k) \chi \right) + \right. \\
&& \left.
+ ( \mu_x - i \mu_y)
\left( \cos ( \kappa' k' + \kappa k) \chi + i \sin ( \kappa' k' + \kappa k) \chi \right) (-1)^{p'+p} (-1)^{K'+K+1}
\right] \\ \nonumber
&& = 
\left( \begin{array}{ccc} 
 J &  1 &  J' \\
-K &  1 & K-1  \end{array} \right)
\frac{(-1)^{J+p+K}}{\sqrt{2}} \\ \nonumber
&& \left[ 
-  \mu_x \cos ( \kappa' k' + \kappa k) \chi \left( 1- (-1)^{p'+p} \right)
-i \mu_y \cos ( \kappa' k' + \kappa k) \chi \left( 1+ (-1)^{p'+p} \right) - \right. \\ 
&& \left.
-  \mu_y \sin ( \kappa' k' + \kappa k) \chi \left( 1- (-1)^{p'+p} \right)
+i \mu_x \sin ( \kappa' k' + \kappa k) \chi \left( 1+ (-1)^{p'+p} \right) \right]
\end{eqnarray}
Since $p' \ne p$ then $(-1)^{p'+p} = -1$ and
\begin{equation}
\{...\} = 
\left( \begin{array}{ccc} 
 J &  1 &  J' \\
-K &  1 & K-1  \end{array} \right)
\frac{(-1)^{J+p+K}}{\sqrt{2}} 2  \left[
- \mu_x \cos ( \kappa' k' + \kappa k) \chi -  \mu_y \sin ( \kappa' k' + \kappa k) \chi \right]
\end{equation}

Term V is not zero only if $K' = -K-1$ and therefore cannot exist.

Term VI exists only if $K' = K +1$  i.e. $K = 0, 1, 2...$ (while $K' = 1, 2, 3...$ respectively)
\begin{eqnarray} \nonumber
&& \{...\} = 
\left( \begin{array}{ccc} 
 J &  1 &  J' \\
 K &  1 & -K-1  \end{array} \right)
\frac{1}{\sqrt{2}} \\ \nonumber
&& \left[ ( -\mu_x - i \mu_y) 
\left( \cos ( -\kappa' k' + \kappa k) \chi + i \sin ( -\kappa' k' + \kappa k) \chi \right) + \right. \\ 
&& \left.
+ ( \mu_x - i \mu_y)
\left( \cos ( -\kappa' k' + \kappa k) \chi - i \sin ( -\kappa' k' + \kappa k) \chi \right) (-1)^{p'+p}
\right] \\ \nonumber
&&  = 
\left( \begin{array}{ccc} 
 J &  1 &  J' \\
 K &  1 & -K-1  \end{array} \right)
\frac{1}{\sqrt{2}} \\ \nonumber
&& \left[ 
-  \mu_x \cos ( -\kappa' k' + \kappa k) \chi \left( 1- (-1)^{p'+p} \right)
-i \mu_y \cos ( -\kappa' k' + \kappa k) \chi \left( 1+ (-1)^{p'+p} \right) + \right. \\ 
&& \left.
+  \mu_y \sin ( -\kappa' k' + \kappa k) \chi \left( 1- (-1)^{p'+p} \right)
-i \mu_x \sin ( -\kappa' k' + \kappa k) \chi \left( 1+ (-1)^{p'+p} \right) \right]
\end{eqnarray}
Since $p' \ne p$ then $(-1)^{p'+p} = -1$ and
\begin{equation}
\{...\} = 
\left( \begin{array}{ccc} 
 J &  1 &  J' \\
 K &  1 & -K-1  \end{array} \right)
\frac{2}{\sqrt{2}} \left[
- \mu_x \cos ( -\kappa' k' + \kappa k) \chi +  \mu_y \sin ( -\kappa' k' + \kappa k) \chi \right]
\end{equation}
The projections of dipole are real functions and therefore $X(J,J')$ is real too.

Let us define $\Delta K = K - K'$, ie $K = K'+\Delta K$.
The equations can be also summarized in a more compact form for different cases of $\Delta K$ and $K'$. \\
$\Delta K = 0$, $K = K'= 0$ (Terms II and I)
\begin{equation}
\{...\} = 
\left( \begin{array}{ccc} 
 J &  1 &  J' \\
 0 &  0 &  0   \end{array} \right)
\mu_z \; 2  \left[ 
\cos ( \kappa' k' - \kappa k) \chi
 + (-1)^{J'+p'} \cos ( \kappa' k' + \kappa k) \chi \right]
\end{equation}
$\Delta K = 0$, $K = K' > 0$ (Term II)
\begin{equation}
\{...\} = 
\left( \begin{array}{ccc} 
 J  &  1 &  J' \\
 K' &  0 & -K'  \end{array} \right)
\mu_z \; 2  \cos ( \kappa' k' - \kappa k) \chi
\end{equation}
$\Delta K = 1$, $K'= 0$, $K= 1$ (Terms III and IV)
\begin{eqnarray} \nonumber
&&\{...\} = 
\left( \begin{array}{ccc} 
 J &  1 &  J' \\
-1 &  1 &  0   \end{array} \right)
\sqrt{2} \;   \\ \nonumber
&& \left[ \mu_x \left(
(-1)^{J'+J+1} \cos ( \kappa' k' - \kappa k) \chi +
(-1)^{J+p} \cos ( \kappa' k' + \kappa k) \chi \right) \right. \\
&& \left. + \mu_y \left(
(-1)^{J'+J} \sin ( \kappa' k' - \kappa k) \chi +
(-1)^{J+p} \sin ( \kappa' k' + \kappa k) \chi \right)
\right]
\end{eqnarray}
$\Delta K = 1$, $K'> 0$, $K=K'+1$ (Term III)
\begin{equation}
\{...\} = 
\left( \begin{array}{ccc} 
 J    &  1 &  J' \\
-K'-1 &  1 &  K'  \end{array} \right)
\sqrt{2} \; \left[ \mu_x  (-1)^{J'+J+1} \cos ( \kappa' k' - \kappa k) \chi 
+ \mu_y  (-1)^{J'+J} \sin ( \kappa' k' - \kappa k) \chi  \right]
\end{equation}
$\Delta K = -1$, $K'= 1$, $K= 0$ (Terms VI and IV)
\begin{eqnarray} \nonumber
&&\{...\} = 
\left( \begin{array}{ccc} 
 J &  1 &  J' \\
 0 &  1 & -1   \end{array} \right)
\sqrt{2} \; \\ \nonumber
&& \left[ \mu_x  \left(
- \cos ( \kappa' k' - \kappa k) \chi +
(-1)^{J+p+1} \cos ( \kappa' k' + \kappa k) \chi \right) \right. \\ 
&& \left. + \mu_y  \left(
- \sin ( \kappa' k' - \kappa k) \chi +
(-1)^{J+p+1} \sin ( \kappa' k' + \kappa k) \chi \right)
\right]
\end{eqnarray}
$\Delta K = -1$, $K'> 1$, $K=K'-1$ (Term VI)
\begin{equation}
\{...\} = 
\left( \begin{array}{ccc} 
 J    &  1 &  J' \\
 K'-1 &  1 & -K'   \end{array} \right)
\sqrt{2} \; \left[ - \mu_x  \cos ( \kappa' k' - \kappa k) \chi 
- \mu_y  \sin ( \kappa' k' - \kappa k) \chi \right]
\end{equation}

The expressions above provided the basis for performing numerical line strength calculations
and were used in the new \textsc{dipole4} code to compute line intensities.
In practice, we calculate $X(J,J')$ as a sum of three terms each depending on either $\mu_x$, $\mu_y$ or $\mu_z$.
The code supplements \textsc{wavr4} \citep{Tennyson} 
and was written in Fortran~90 as \textsc{wavr4} itself.
Although the formulae above are general, currently only the case of $J=0$ 
was fully implemented.
The code takes as an input a set of initial and final eigenfunctions, a dipole function,
and a control file and computes line strengths between all possible combinations of 
initial and final states.
It follows from the equations above that there must be necessarily a sum (i.e a loop) over
all angular and radial basis functions and an integration of the dipole function
in order to obtain its matrix representation in the angular basis.
We should note again that we assume diagonal representation of the dipole function
in the radial basis.
Since the basis functions (both FBR and DVR parts) are the same as those
used in \textsc{wavr4}, we could readily use the integration functions from \textsc{wavr4} 
for integration over the angular (FBR) basis and perform
direct summation for the integration over the radial grids.
For the purposes of the current project we only needed evaluation of matrix elements of 
the dipole moment function and \textsc{dipole4} code has been modified accordingly.

One note on the implementation is in order.
If the integration-summation formulae are taken literally then
they may result in matrix vector multiplication which is very
inefficient on cache-based computer architectures for matrices larger than cache.
Therefore we decided to operate simultaneously on several initial and final eigenfunctions
and thus utilise matrix-matrix multiplication which is much more efficient.
At the core, the new code is essentially a loop over the three-dimensional radial grid 
which builds a matrix representation of the integrand in the primitive basis 
and performs computations of integrals involving all possible initial and final states.  
The intermediate results are accumulated in a matrix of the size equal to the number 
of initial times final states.
The loop over the radial grid lends itself to a very efficient parallelisation strategy.
In the parallel implementation, the accumulators reside on every
parallel processor so that no communication is necessary during the computation.
Only a final reduction is required to gather the intermediate accumulators 
and compute the transition matrix.


Finally, one additional step is required for the calculation of dipole transitions.
The wave-functions produced by \textsc{wavr4} need to be transformed
back in to the primitive basis on which \textsc{dipole4} operates.
This is done using a separate program and the results are stored on disk. 
The two step approach allows more efficient calculation of the integrals, 
reuse of converted wave functions as well as potential use of \textsc{dipole4} 
with the codes which do not implement contraction-reduction strategy 
(e.g. direct methods).

\begin{verbatim}
%  structure of the dipole matrix  p' /= p
%
%        K=0            K=1           K=2
%        ____________  ____________  ____________
%  K'=0 [ (II)  mu_z ][ (III) mu_x ][            ][
%       [ (I)   mu_z ][ (IV)  mu_y ][            ][
%       [____________][____________][____________][
%  K'=1 [ (VI)  mu_x ][ (II)  mu_z ][ (III) mu_x ][
%       [ (IV)  mu_y ][            ][       mu_y ][
%       [____________][____________][____________][ \
%  K'=2 [            ][ (VI)  mu_x ][ (II)  mu_z ][   \ dK = 1
%       [            ][       mu_y ][            ][
%       [____________][____________][____________][
%                            \                 \ 
%  dK = K - K',  K=K'+dK       \ dK = -1         \ dK = 0
\end{verbatim}

